%!TEX TS-program = xelatex
\documentclass[12pt, a4paper, oneside]{article}

% Можно вставить разную преамбулу
% пакеты для математики
\usepackage{amsmath,amsfonts,amssymb,amsthm,mathtools}  
\mathtoolsset{showonlyrefs=true}  % Показывать номера только у тех формул, на которые есть \eqref{} в тексте.

\usepackage[british,russian]{babel} % выбор языка для документа
\usepackage[utf8]{inputenc}          % utf8 кодировка

% Основные шрифты 
\usepackage{fontspec}         
\setmainfont{Linux Libertine O}  % задаёт основной шрифт документа

% Математические шрифты 
\usepackage{unicode-math}     
\setmathfont[math-style=upright]{euler.otf} 

\setmathfont[range={\mathbb, \mathop, \heartsuit, \angle, \smile, \varheartsuit}]{Asana-Math.otf}

%%%%%%%%%% Работа с картинками и таблицами %%%%%%%%%%
\usepackage{graphicx} % Для вставки рисунков                
\usepackage{graphics}
\graphicspath{{images/}{pictures/}}   % папки с картинками

\usepackage[figurename=Картинка]{caption}

\usepackage{wrapfig}    % обтекание рисунков и таблиц текстом

\usepackage{booktabs}   % таблицы как в годных книгах
\usepackage{tabularx}   % новые типы колонок
\usepackage{tabulary}   % и ещё новые типы колонок
\usepackage{float}      % возможность позиционировать объекты в нужном месте
\renewcommand{\arraystretch}{1.2}  % больше расстояние между строками


%%%%%%%%%% Графики и рисование %%%%%%%%%%
\usepackage{tikz, pgfplots}  % языки для графики
%\pgfplotsset{compat=1.16}

\usepackage{todonotes} % для вставки в документ заметок о том, что осталось сделать
% \todo{Здесь надо коэффициенты исправить}
% \missingfigure{Здесь будет Последний день Помпеи}
% \listoftodos --- печатает все поставленные \todo'шки

\usepackage{multicol}

%%%%%%%%%% Внешний вид страницы %%%%%%%%%%

\usepackage[paper=a4paper, top=20mm, bottom=15mm,left=20mm,right=15mm]{geometry}
\usepackage{indentfirst}    % установка отступа в первом абзаце главы

\usepackage{setspace}
\setstretch{1.15}  % межстрочный интервал
\setlength{\parskip}{4mm}   % Расстояние между абзацами
% Разные длины в LaTeX: https://en.wikibooks.org/wiki/LaTeX/Lengths

% свешиваем пунктуацию
% теперь знаки пунктуации могут вылезать за правую границу текста, при этом текст выглядит ровнее
\usepackage{microtype}

% \flushbottom                            % Эта команда заставляет LaTeX чуть растягивать строки, чтобы получить идеально прямоугольную страницу
\righthyphenmin=2                       % Разрешение переноса двух и более символов
\widowpenalty=300                     % Небольшое наказание за вдовствующую строку (одна строка абзаца на этой странице, остальное --- на следующей)
\clubpenalty=3000                     % Приличное наказание за сиротствующую строку (омерзительно висящая одинокая строка в начале страницы)
\tolerance=10000     % Ещё какое-то наказание.

% мои цвета https://www.artlebedev.ru/colors/
\definecolor{titleblue}{rgb}{0.2,0.4,0.6} 
\definecolor{blue}{rgb}{0.2,0.4,0.6} 
%\definecolor{red}{rgb}{1,0,0.2} 
\definecolor{green}{rgb}{0, 0.6, 0}
\definecolor{purp}{rgb}{0.4,0,0.8} 

\definecolor{red}{RGB}{213,94,0}
\definecolor{yellow}{RGB}{240,228,66}


% цвета из geogebra 
\definecolor{litebrown}{rgb}{0.6,0.2,0}
\definecolor{darkbrown}{rgb}{0.75,0.75,0.75}

% Гиперссылки
\usepackage{xcolor}   % разные цвета

\usepackage{hyperref}
\hypersetup{
	unicode=true,           % позволяет использовать юникодные символы
	colorlinks=true,       	% true - цветные ссылки
	urlcolor=blue,          % цвет ссылки на url
	linkcolor=black,          % внутренние ссылки
	citecolor=green,        % на библиографию
	breaklinks              % если ссылка не умещается в одну строку, разбивать её на две части?
}

% меняю оформление секций 
\usepackage{titlesec}
\usepackage{sectsty}

% меняю цвет на синий
\sectionfont{\color{titleblue}}
\subsectionfont{\color{titleblue}}

% кружочки у цифр в секциях
\renewcommand{\thesection}{\arabic{section}}

% https://ru.overleaf.com/learn/latex/Sections_and_chapters

% выбрасываю нумерацию страниц и колонтитулы 
%\pagestyle{empty}

% синие круглые бульпоинты в списках itemize 
\usepackage{enumitem}

\definecolor{itemizeblue}{rgb}{0, 0.45, 0.70}

\newcommand*{\MyPoint}{\tikz \draw [baseline, fill=itemizeblue, draw=blue] circle (2.5pt);}
\renewcommand{\labelitemi}{\MyPoint}

\AddEnumerateCounter{\asbuk}{\@asbuk}{\cyrm}
\renewcommand{\theenumi}{\asbuk{enumi}}

% расстояние в списках
\setlist[itemize]{parsep=0.4em,itemsep=0em,topsep=0ex}
\setlist[enumerate]{parsep=0.4em,itemsep=0em,topsep=0ex}

% эпиграфы
\usepackage{epigraph}
\setlength\epigraphwidth{.6\textwidth}
\setlength\epigraphrule{0pt}

%%%%%%%%%% Свои команды %%%%%%%%%%

% Математические операторы первой необходимости:
\DeclareMathOperator{\sgn}{sign}
\DeclareMathOperator*{\argmin}{arg\,min}
\DeclareMathOperator*{\argmax}{arg\,max}
\DeclareMathOperator{\Cov}{Cov}
\DeclareMathOperator{\Var}{Var}
\DeclareMathOperator{\Corr}{Corr}

\DeclareMathOperator{\Pois}{Pois}
\DeclareMathOperator{\Geom}{Geom}
\DeclareMathOperator{\Exp}{Exp}

%\DeclareMathOperator{\E}{\mathbb{E}}
\DeclareMathOperator{\Med}{Med}
\DeclareMathOperator{\Mod}{Mod}
\DeclareMathOperator*{\plim}{plim}

% команды пореже
\newcommand{\const}{\mathrm{const}}  % const прямым начертанием
\newcommand{\iid}{\sim i\,i\,d\,\,}  % ну вы поняли...
\newcommand{\fr}[2]{\ensuremath{^{#1}/_{#2}}}   % особая дробь
\newcommand{\ind}[1]{\mathbbm{1}_{\{#1\}}} % Индикатор события
\newcommand{\dx}[1]{\,\mathrm{d}#1} % для интеграла: маленький отступ и прямая d

% одеваем шапки на частые штуки
\def \hb{\hat{\beta}}
\def \hs{\hat{s}}
\def \hy{\hat{y}}
\def \hY{\hat{Y}}
\def \he{\hat{\varepsilon}}
\def \hVar{\widehat{\Var}}
\def \hCorr{\widehat{\Corr}}
\def \hCov{\widehat{\Cov}}

% Греческие буквы
\def \a{\alpha}
\def \b{\beta}
\def \t{\tau}
\def \dt{\delta}
\def \e{\varepsilon}
\def \ga{\gamma}
\def \kp{\varkappa}
\def \la{\lambda}
\def \sg{\sigma}
\def \tt{\theta}
\def \Dt{\Delta}
\def \La{\Lambda}
\def \Sg{\Sigma}
\def \Tt{\Theta}
\def \Om{\Omega}
\def \om{\omega}

% Готика
\def \mA{\mathcal{A}}
\def \mB{\mathcal{B}}
\def \mC{\mathcal{C}}
\def \mE{\mathcal{E}}
\def \mF{\mathcal{F}}
\def \mH{\mathcal{H}}
\def \mL{\mathcal{L}}
\def \mN{\mathcal{N}}
\def \mU{\mathcal{U}}
\def \mV{\mathcal{V}}
\def \mW{\mathcal{W}}

% Жирные буквы
\def \mbb{\mathbb}
\def \RR{\mbb R}
\def \NN{\mbb N}
\def \ZZ{\mbb Z}
\def \PP{\mbb{P}}
\def \E{\mbb{E}}
\def \QQ{\mbb Q}

\def\F{\ensuremath{\mathcal{F}}} % аналогично!

%%%%%%%%%% Теоремы %%%%%%%%%%
\theoremstyle{plain} % Это стиль по умолчанию.  Есть другие стили.
\newtheorem{theorem}{Теорема}[section]
\newtheorem{proposition}{Утверждение}[section]
\newtheorem{result}{Следствие}[section]

% убирает курсив и что-то еще наверное делает ;)
\theoremstyle{definition}         
\newtheorem*{definition}{Определение}  % нумерация не идёт вообще


%%%%%%%%%% Задачки и решения %%%%%%%%%%
\usepackage{etoolbox}    % логические операторы для своих макросов
\usepackage{environ}
\newtoggle{lecture}

\newcounter{probNum}[section]  % счётчик для упражнений 
\NewEnviron{problem}[1]{%
    \refstepcounter{probNum}% увеличели номер на 1 
    {\noindent \textbf{\large \color{titleblue} Упражнение~\theprobNum~#1}  \\ \\ \BODY}
    {}%
  }

% Окружение, чтобы можно было убирать решения из pdf
\NewEnviron{sol}{%
  \iftoggle{lecture}
    {\noindent \textbf{\large Решение:} \\ \\ \BODY}
    {}%
  }
 
% выделение по тексту важных вещей
\newcommand{\indef}[1]{\textbf{ \color{green} #1}} 

% разные дополнения для картинок
\usetikzlibrary{arrows.meta}
\usepackage{varwidth}

\usepackage[normalem]{ulem}  % для зачекивания текста

% Если переключить в false, все solution исчезнут из pdf
\toggletrue{lecture}
%\togglefalse{lecture}



\title{
\begin{center} 
\includegraphics[width=0.99\textwidth]{logo.png}
\end{center}

Посиделка 4: мощь средних}
\date{ } %\today}

% Если делаешь конспект, вписывай своё имя прямо сюда!
\author{Ульянкин Ппилиф \thanks{\url{https://github.com/FUlyankin/matstat_lec}}}

\begin{document} % Конец преамбулы, начало файла

\maketitle

\epigraph{\hspace{5cm} Бесконечность --- не предел!}{\textit{Баз Лайтер, История игрушек (1995)}}

Когда мы обсуждали в первой посиделке схему матстата, мы сказали, что существует несколько разных подходов к построению моделей. У каждого из них есть своя изюминка. В этой посиделке мы попробуем построить матстат на базе среднего. 

\begin{center}
    \begin{tikzpicture}[scale = 1.5, line cap=round,line join=round,x=1.0cm,y=1.0cm]
    
        \path (-1,0) node(x) {\Huge $\hat \theta$} (3,0) node(y) {\Huge $f_{\hat \theta} (t)$};
        
        \node at (8, 0) {\footnotesize
            \begin{varwidth}{\linewidth}\begin{itemize}
                \item прогнозы
                \item насколько точны прогнозы
                \item ответы на вопросы (гипотезы)
            \end{itemize}\end{varwidth}
        };
        
        \draw[->, line width=1.1pt] (-0.5,0) -- (2,0);
        \draw[->, line width=1.1pt] (4,0) -- (6,0);
    
        \node at (-0.2, -1.45) { \footnotesize
            \begin{varwidth}{\linewidth} \textbf{Cоюзники:}
            \begin{itemize}
                \item метод моментов 
                \item обобщённый метод моментов
            \end{itemize}\end{varwidth}
        };
        
        \node at (4.5, -1.45) { \footnotesize
            \begin{varwidth}{\linewidth} \textbf{Cоюзники:}
            \begin{itemize}
                \item центральная предельная теорема (ЦПТ)
                \item Дельта-метод
            \end{itemize}\end{varwidth}
        };
    \end{tikzpicture}
\end{center} 

Искать точечные оценки мы будем с помощью метода моментов, который основан на законе больших чисел (ЗБЧ). Искать распределение оценки нам поможет центральная предельная теорема (ЦПТ). Ближе к концу посиделки мы обобщим ЦПТ до дельта-метода, а метод моментов до обобщённого метода моментов. 

\section{Метод моментов и ЗБЧ}

Пусть случайные величины $X_{1}, \ldots, X_{n}$ независимы и одинаково распределены. Закон больших чисел говорит нам, что среднее выборочное $\bar{X}$ является хорошей оценкой для математического ожидания $ \E(X_{i}) $.

\begin{theorem}{\textbf{Закон Больших Чисел (Пафнутий Львович Чебышёв)}}

Пусть $X_1, \ldots, X_n$ попарно независимые и одинаково распределённые случайные величины с конечным вторым моментом, $0 < E(X_i^2) < \infty$, тогда:

$$
\bar{X}_{n} = \frac{X_1 + \ldots + X_n}{n} \stackrel{p}{\longrightarrow} E(X_1)
$$
\end{theorem}

На практике это означает, что при больших $n$ эти величины примерно равны:

\[
\bar{X}_{n} \approx \E(X_{i}).
\]

На этой нехитрой идее и построен метод моментов. Как конкретно используется идея, понятно из следующих нескольких примеров.

\begin{problem}{(киндеры)}
    Максим любит киндеры и собирает коллекцию пляжных бегемотиков. Для этого он покупает шоколадки. Пусть $p$ --- вероятность того, что в киндере лежит пляжный бегемотик. Максим покупает яйца и записывает номер попытки, с которой у него появилась правильная игрушка.
    
    До первого бегемотика Максим купил $X_1$ яиц. После счётчик обнулился. До второго бегемотика Макс купил $X_2$ яиц и так далее. Всего Макс собрал $n$ наблюдений. Постройте оценку неизвестного параметра $p$ с помощью метода моментов.
\end{problem}

\begin{sol}
\indef{Эксперимент} состоит в том, что Максим ест шоколад и собирает \indef{выборку} из своих попыток раздобыть нужную игрушку. \indef{Вопрос} Макса заключается в том, насколько часто встречаются бегемотики. \indef{Модель} Макса, в которую он верит заключается в том, что величины $X_{i}$ имеют геометрическое распределение, поэтому $\E(X_{i})=\frac{1}{p}$. Принцип метода моментов гласит:
	\[ \bar{X}_{n}\approx \frac{1}{p}.\]
	Выражаем неизвестный параметр $ p $:
	\[ p\approx \frac{1}{\bar{X}_{n}}. \]
	Это и есть нужная нам оценка:
	\[ \hat{p}_{MM} = \frac{1}{\bar{X}_{n}}. \]
\end{sol}


\begin{problem}{(Шарик и фарш)}
Продавщица Глафира отдаёт псу Шарику в конце каждого дня нерасфасованные остатки мясного фарша. Фарш фасуется упаковками по $a$ грамм, поэтому нерасфасованный остаток в $i$-ый день, $X_i$, случаен и равномерно распределен на отрезке $[0;a]$. Пёс Шарик хорошо помнит все $X_1$, \ldots, $X_n$. Помогите псу Шарику найти оценку $a$ методом моментов.
\end{problem}

\begin{sol}
\indef{Эксперимент} состоит в том, что Шарик каждый день ест фарш и собирает \indef{выборку.} \indef{Вопрос} Шарика заключается в том, сколько максимально фарша он может получить. \indef{Модель} Шарика, в которую он верит --- выборка приходит из равномерно распределения. В рамках этой веры мы пробуем решить задачу.

В данном случае $ \E(X_{i})=\frac{a}{2} $ и, следовательно
\[ \bar{X}_{n}\approx \frac{a}{2}. \]
Выражаем $a$
\[ a \approx 2 \cdot \bar{X}_{n}, \]
это и есть нужная нам оценка:
\[ \hat{a}^{MM}= 2 \cdot \bar{X}_{n}. \]
\end{sol}

\begin{definition} 
Пусть $X_1, \ldots, X_n$ одинаково распределены и независимы, а $\E(X_{i})$ зависит от неизвестного параметра $\theta$, скажем $\E(X_{i}) = f(\theta)$. Тогда \indef{оценкой метода моментов} называется случайная величина:
\[ \hat{\theta}_{MM} = f^{-1}(\bar{X}_{n}) \]
\end{definition} 

Конечно, иногда бывают ситуации, когда математическое ожидание $\E(X_{i})$ не зависит от $\theta$. Например, если $X_{i}$ равномерны на $[-\theta; \theta]$, то математическое ожидание $\E(X_{i}) = 0$. Что делать в такой ситуации? 

Неспроста же наш метод называется методом моментов\ldots{}  Напомним, что $k$-ым моментом случайной величины $X_{i}$ называется математическое ожидание $\E(X_{i}^{k})$. 

Если условия $\bar{X}_{n}\approx \E(X_{i}),$ связанного с первым моментом не хватило, то на помощь придет второй момент случайной величины. В силу того же закона больших чисел:

\[\overline{X^2}_n = \frac{X_{1}^{2} + \ldots + X_{n}^{2}}{n} \approx \E(X_{i}^{2})\]

\begin{problem}{(равномерное)}
Величины $ X_{i} $ независимы и равномерны на $ [-\theta;\theta] $. Постройте оценку неизвестного параметра $ \theta $ с помощью метода моментов.
\end{problem}

\begin{sol}
Убеждаемся, что $\E(X_{i})=0$:

\[  \E(X_{i}) = \int _{-\theta}^{\theta} x \cdot \frac{1}{2\theta} \dx{x} = \left. \frac{x^2}{4 \theta} \right|_{-\theta}^{\theta} = \left( \frac{\theta^2 - (-\theta)^2 }{4 \theta}  \right) =  \frac{\theta^{2} - \theta^{2}}{4 \theta} = 0  \]

Находим $ \E(X_{i}^{2}) $:

\[  \E(X_{i}^{2}) = \int _{-\theta}^{\theta} x^{2} \cdot \frac{1}{2\theta}  \dx{x} = \left. \frac{x^3}{6 \theta} \right|_{-\theta}^{\theta} = \left( \frac{\theta^3 - (-\theta)^3 }{6\theta}  \right) = \frac{2 \theta^{3}}{6\theta}  = \frac{\theta^{2}}{3}  \]

Согласно принципу метода моментов

\[ \frac{\sum X_{i}^{2}}{n} \approx \frac{\theta^{2}}{3}.\]

Выражаем $ \theta $

\[ \theta\approx \sqrt{3\frac{\sum X_{i}^{2}}{n} }.\]

Это и есть нужная нам оценка

\[ \hat{\theta}_{MM}= \sqrt{3\frac{\sum X_{i}^{2}}{n} } = \sqrt{3 \overline{X^2 } }.\]
\end{sol}

Если не хватит и второго момента, тогда воспользуемся третьим и т.д. Для произвольного $k$ мы имеем

\[ \frac{X_{1}^{k} + \ldots + X_{n}^{k}}{n} \approx \E(X_{i}^{k})\]

В большинстве случаев хватает именного первого момента. Последующие моменты нужны чаще всего при оценке нескольких параметров.

\begin{problem}{(Нормальное)}
Мстислав собрал выборку $X_1, \ldots, X_n$ из нормального распределения $\mN(\mu, \sigma^2)$ и теперь хочет методом моментов оценить неизвестные параметры. Нужно ему помочь в этом нелёгком деле.
\end{problem}

\begin{sol}
У нас два параметра. Будем использовать два уравнения

\[
\begin{cases} 
\E(X^2) \approx \overline{X^2}_n   \\
\E(X^2) \approx  \bar{X}_n 
\end{cases} \Leftrightarrow \begin{cases} 
\sigma^2 + \mu \approx \overline{X^2}_n   \\
 \mu \approx \bar{X}_n
\end{cases}  \Leftrightarrow \begin{cases} 
 \hat{\sigma}^2 = \overline{X^2}_n - \hat{\mu} = \overline{X^2}_n - \bar{X}_n \\
\hat{\mu} = \bar{X}_n.
\end{cases}
\]

Обратите внимание, что колпачок над параметрами появляется в тот момент, когда мы строго приравниваем математическое ожидание к среднему. Начиная с этого момента мы имеем дело с оценками. Для реального распределение значение математических ожиданий может отличаться от средних, посчитанных по конкретным выборкам. 
\end{sol} 


\section{Метод моментов и ЦПТ}

Точечная оценка --- это хорошо. К сожалению, чаще всего нам этого мало. Всегда хочется понимать, насколько оценка получилась точной. В рамках статистики средних, нам в этом помогают \indef{доверительные интервалы,} то есть промежутки, которые накрывают истинное значение параметра с высокой вероятностью. Построить доверительные интервалы нам помогает ЦПТ.

\begin{theorem}{\textbf{Центральная Предельная Теорема (Прокопий Петрович Ляпунов)}}

Пусть $X_1, \ldots, X_n$ попарно независимые и одинаково распределённые случайные величины с конечным вторым моментом, $0 < E(X_i^2) < \infty$, тогда при $n \to \infty$ имеет место сходимость по распределению: 

\[
\bar{X}_n \stackrel{d}{\longrightarrow} N \left(\E(X_i), \frac{\Var(X_i)}{n} \right)
\]

Этот же факт можно переписать немного иначе

\[
\sqrt{n} \cdot [\bar{x} - \E(X_i)]  \stackrel{d}{\longrightarrow} N \left(0, \Var(X_i)\right)
\]

или даже 

\[
\sqrt{n} \cdot \frac{\bar{x} - \E(X_i)}{\sqrt{\Var(X_i)}}  \stackrel{d}{\longrightarrow} N \left(0, 1\right).
\]
\end{theorem}

Если говорить простым языком, то при определённых условиях сумма достаточно большого числа случайных величин имеет распределение близкое к нормальному. \indef{Главное, чтобы случайные величины были похожи и не было такого, что одна резко выделяется на фоне остальных.} Иначе мы окажемся в Крайнеземье, где работают совсем другие законы. Давайте посмотрим, как будет вести себя среднее в примере с Шариком и Глафирой.

\begin{problem}{(Шарик и фарш)}
Величины $X_1$, \ldots, $X_n \iid U[0;a]$. Найдите для $a$ оценку методом моментов и постройте для неё доверительный интервал. У Шарика есть гипотеза, что вес упаковки не может превышать $100$ грамм. Формализуйте эту гипотезу и опишите процедуру её проверки. 
\end{problem}

\begin{sol}
Мы нашли оценку метода моментов, оказалось что 
\[ \hat{a} = 2 \cdot \bar{X}_{n}. \]

В данном случае мы работаем с равномерным распределением, а значит $\Var(X_i) = \frac{a^2}{12}.$ Получается, что $\Var(\bar x) = \frac{a^2}{12 \cdot n}.$ Воспользовавшись ЦПТ мы можем выписать асимптотическое распределение среднего 

\[\bar{X}_n \overset{\text{\textit{asy}}}{\sim} \mN \left(\frac{a}{2}, \frac{a^2}{12 \cdot n} \right).\] 

По свойствам нормального распределения, получаем распределение для оценки неизвестного параметра

\[\hat{a}= 2 \bar{X}_{n} \overset{\text{\textit{asy}}}{\sim} \mN \left(a, \frac{a^2}{3 \cdot n} \right).\] 

Математическим ожиданием нашей оценки является сам параметр. Это говорит о том, что оценка получилась несмещённой. Нам хочется понять, насколько оценка точная. Для этого нужно знать дисперсию. В ней фигурирует неизвестное значение $a$. Его смело можно заменить на оценку $\hat{a}$. Асимптотика от этого никак не испортится. Чуть позже мы это докажем. 

\[\hat{a}= 2 \bar{X}_{n} \overset{\text{\textit{asy}}}{\sim} \mN \left(a, \frac{4 \bar{X}^2}{3 \cdot n} \right).\] 

Немного перепишем 

\[Z = \frac{ 2 \bar{X}_{n} - a}{\sqrt{\frac{4 \bar{X}^2}{3 \cdot n}} } \overset{\text{\textit{asy}}}{\sim} \mN \left(0, 1 \right).\] 

Попробуем понять насколько точной получилась наша оценка. Зажмём случайную величину $Z$ между её квантилями так, чтобы 

$$
\PP \left( z_{1 - \frac{\alpha}{2}} \le \frac{ 2 \bar{X}_{n} - a}{\sqrt{\frac{4 \bar{X}^2}{3 \cdot n}} } \le z_{1 - \frac{\alpha}{2}} \right) = 1 - \alpha.
$$

Такой интервал для случайной величины $Z$ называется \indef{предиктивным интервалом.} Его границы фиксированы, а в центре находится случайная величина. Давайте разрешим неравенство относительно $a$. Тогда мы получим, что

$$
\PP\left(2 \bar{X}_{n}  - z_{1 - \frac{\alpha}{2}} \cdot \sqrt{\frac{4 \bar{X}^2}{3 \cdot n}}  \le a \le 2 \bar{X}_{n}  + z_{1 - \frac{\alpha}{2}} \cdot \sqrt{\frac{4 \bar{X}^2}{3 \cdot n}} \right) = 1 - \alpha.
$$

Такой интервал называется \indef{доверительным интервалом.} Его границы --- случайные величины, а в середине стоит неизвестная константа, которую доверительный интервал накрывает с вероятностью $1 - \alpha$. Если мы соберём много-много выборок и построим по каждой доверительный интервал для $a$, он в $\alpha$ ситуациях окажется за пределами доверительного интервала. Здесь $\alpha$ это уровень значимости. Тот же самый, о котором мы говорили в первой посиделке. 


Займёмся гипотезой шарика. Ему кажется, что вес упаковки не превышает $100$ грамм. Для начала \indef{проверим гипотезу о том, что вес упаковки не отличается от $100$ грамм}

\begin{equation*} 
\begin{aligned} 
& H_0: a = 100 \\
& H_a: a \ne 100.
\end{aligned} 
\end{equation*} 

Оценка $\hat a$ --- случайная величина. Расстояние $\hat a - 100$ тоже случайная величина. Если наша гипотеза верна, расстояние $\hat a - 100$ должно быть близко к нулю. 

\todo[inline]{Дописать оба варианта проверки гипотезы, сделать отсылки к 1 лекции.}



\pgfplotsset{
  myplot/.style={
    width = 12cm, height = 6cm,
    xlabel = $t$, ylabel = $f(t)$,
    samples = 75,
    domain = -5:5,
    xlabel style = {at = {(1,0)}, anchor = west},
    ylabel style = {rotate = -90, at = {(0, 1)}, anchor = south west},
    legend style = {draw = none, fill = none},
  }
}

\begin{tikzpicture}[>=stealth,
  every node/.style={rounded corners},
  declare function={
    gamma(\z)=
    (2.506628274631*sqrt(1/\z)+0.20888568*(1/\z)^(1.5)+
    0.00870357*(1/\z)^(2.5)-(174.2106599*(1/\z)^(3.5))/25920-
    (715.6423511*(1/\z)^(4.5))/1244160)*exp((-ln(1/\z)-1)*\z);
  },
  declare function={
    student(\x,\n)=
    gamma((\n+1)/2)/(sqrt(\n*pi)*
    gamma(\n/2))*((1+(\x*\x)/\n)^(-(\n+1)/2));
  }]

  \begin{axis}[myplot, smooth]

  \foreach \zValue/\al/\pos in {2.145/0.025/0.30} {
    \addplot[domain = -4:-\zValue, draw = none, fill = cyan, opacity = 0.5] {student(x, 14)} \closedcycle;
    \addplot[domain = \zValue:4, draw = none, fill = orange, opacity = 0.5] {student(x, 14)} \closedcycle;
    \edef\temp{\noexpand
      \path[<->, draw] (axis cs: -\zValue, 0) to[out = 90, in = 0]
        (axis cs: -\zValue + 0.2, \pos) node[left] {$-t_{\al} = -\zValue$};
    }
    \temp
    \edef\temp{\noexpand
      \path[<->, draw] (axis cs: \zValue, 0) to[out = 90, in = 180]
        (axis cs: \zValue - 0.2, \pos) node[right] {$t_{\al} = \zValue$};
    }
    \temp
  }

  \addplot[smooth, thick, domain = -4:4, color = gray] {student(x, 14)}
    node[pos = 0.54, pin = {right:$\nu = 15 - 1$}] {};

  \end{axis}
\end{tikzpicture}%













\pgfplotsset{
  myplot/.style={
    width = 12cm, height = 6cm,
    xlabel = $z$, ylabel = $f(z)$,
    samples = 75,
    xlabel style = {at = {(1,0)}, anchor = west},
    ylabel style = {rotate = -90, at = {(0, 1)}, anchor = south west},
    legend style = {draw = none, fill = none},
  }
}

\begin{tikzpicture}[>=stealth,
  every node/.style={rounded corners},
  declare function={
    normalpdf(\x,\mu,\sigma)=
    (2*3.1415*\sigma^2)^(-0.5)*exp(-(\x-\mu)^2/(2*\sigma^2));
  }]

  \begin{axis}[myplot, smooth]

  \foreach \zValue/\al/\pos in {1.645/0.05/0.25, 1.960/0.025/0.20, 2.576/0.005/0.15} {
    \addplot[domain = -4:-\zValue, draw = none, fill = cyan, opacity = 0.15] {normalpdf(x, 0, 1)} \closedcycle;
    \addplot[domain = \zValue:4, draw = none, fill = orange, opacity = 0.15] {normalpdf(x, 0, 1)} \closedcycle;
    \edef\temp{\noexpand
      \path[<->, draw] (axis cs: -\zValue, 0) to[out = 90, in = 0]
        (axis cs: -2, \pos) node[left] {$-z_{\al} = -\zValue$};
    }
    \temp
    \edef\temp{\noexpand
      \path[<->, draw] (axis cs: \zValue, 0) to[out = 90, in = 180]
        (axis cs: 2, \pos) node[right] {$z_{\al} = \zValue$};
    }
    \temp
  }

  \addplot[smooth, thick, domain = -4:4, color = gray] {normalpdf(x,0,1)};

  \end{axis}
\end{tikzpicture}%



\begin{tikzpicture}[
  >=stealth,
  declare function={
    gamma(\z)=
    (2.506628274631*sqrt(1/\z)+0.20888568*(1/\z)^(1.5)+
    0.00870357*(1/\z)^(2.5)-(174.2106599*(1/\z)^(3.5))/25920-
    (715.6423511*(1/\z)^(4.5))/1244160)*exp((-ln(1/\z)-1)*\z);
  },
  declare function={
    student(\x,\n)=
    gamma((\n+1)/2)/(sqrt(\n*pi)*
    gamma(\n/2))*((1+(\x*\x)/\n)^(-(\n+1)/2));
  },
  declare function={
    normalpdf(\x,\mu,\sigma)=
    (2*3.1415*\sigma^2)^(-0.5)*exp(-(\x-\mu)^2/(2*\sigma^2));
  }]

  \begin{axis}[
    width=8cm, height=5cm,
    samples=30,
    xlabel=$t$, ylabel=$f(t)$,
    xlabel style={at={(1,0)}, anchor=north west},
    ylabel style={rotate=-90, at={(0,1)}, anchor=south east},
    legend style={draw=none, fill=none},
    domain=-5:5, xmin=-5.5, xmax=5.5]

    %% Quantil para \nu=5 e \alpha=0.05
    \addplot[domain=2.015:5, draw=none, fill=darkgreen, samples=30]
    {student(x,5)} \closedcycle;

    \addplot[smooth, thick, samples=100] {student(x,5)}
    node[pos=0.55, anchor=mid west, xshift=2em,
    append after command={
      (\tikzlastnode.west) edge [thin, gray] +(-2em,0)}]
    {$\nu=5$};

    \path[<->, draw] (axis cs: 2.015, 0.0) to[out=90, in=-90]
    (axis description cs: 0.82, 0.3) node[above] {$t_{\alpha}=2.0150$};

  \end{axis}
\end{tikzpicture}




\pgfplotsset{
  myplot/.style = {
    width = 10cm, height = 5cm,
    samples = 75,
    ticks = none,
  }
} %

\begin{tikzpicture}[>=stealth,
  every node/.style={rounded corners},
  declare function={
    normalpdf(\x,\mu,\sigma)=
    (2*3.1415*\sigma^2)^(-0.5)*exp(-(\x-\mu)^2/(2*\sigma^2));
  }]

  \begin{scope}
    \begin{axis}[myplot, smooth]
      \def\zValue{1.645}
      \addplot[domain = -4:-\zValue, draw = none, fill = cyan, opacity = 0.75] {normalpdf(x, 0, 1)} \closedcycle;
      \addplot[domain = \zValue:4, draw = none, fill = orange, opacity = 0.75] {normalpdf(x, 0, 1)} \closedcycle;
      \addplot[smooth, thick, domain = -4:4, color = gray] {normalpdf(x,0,1)};
      \path[draw, o->] (axis cs: 2, 0.02) to[out = 90, in = 180] (axis cs: 3, 0.2) node[right] {$\alpha/2$};
      \path[draw, o->] (axis cs: -2, 0.02) to[out = 90, in = 0] (axis cs: -3, 0.2) node[left] {$\alpha/2$};
      \node at (axis cs: 0, 0.1) {$1 - \alpha$};
      \node[below right, align = left] at (axis description cs: 0.01, 0.97) {Bilateral\\ $H_0: \theta \neq \theta_0$};
    \end{axis}
  \end{scope}

  \begin{scope}[yshift = -3.75cm]
    \begin{axis}[myplot, smooth]
      \def\zValue{1.282}
      \addplot[domain = -4:-\zValue, draw = none, fill = cyan, opacity = 0.75] {normalpdf(x, 0, 1)} \closedcycle;
      \addplot[smooth, thick, domain = -4:4, color = gray] {normalpdf(x,0,1)};
      \path[draw, o->] (axis cs: -1.7, 0.03) to[out = 90, in = 0] (axis cs: -3, 0.2) node[left] {$\alpha$};
      \node at (axis cs: 0.25, 0.1) {$1 - \alpha$};
      \node[below right, align = left] at (axis description cs: 0.01, 0.97) {Unilateral {\`a} esqueda\\ $H_0: \theta < \theta_0$};
    \end{axis}
  \end{scope}

  \begin{scope}[yshift = -7.5cm]
    \begin{axis}[myplot, smooth]
      \def\zValue{1.282}
      \addplot[domain = \zValue:4, draw = none, fill = orange, opacity = 0.75] {normalpdf(x, 0, 1)} \closedcycle;
      \addplot[smooth, thick, domain = -4:4, color = gray] {normalpdf(x,0,1)};
      \path[draw, o->] (axis cs: 1.7, 0.03) to[out = 90, in = 180] (axis cs: 3, 0.2) node[right] {$\alpha$};
      \node at (axis cs: -0.25, 0.1) {$1 - \alpha$};
      \node[below right, align = left] at (axis description cs: 0.01, 0.97) {Unilateral {\`a} direita\\ $H_0: \theta > \theta_0$};
    \end{axis}
  \end{scope}

\end{tikzpicture}%------------------------------------------------------







\end{sol}

Когда оценка моментов в явном виде выражается через среднее, работать с ним довольно легко. Давайте посмотрим на задачу про бегемотиков. Там получилось, что $\hat{p} = \frac{1}{\bar{X}_n}.$ Здесь ЦПТ влоб использовать не получится. Зато мы можем воспользоваться её обобщением, \indef{дельта-методом.}


\section{Дельта-метод}

Нормальное распределение возникает, если суммируется большое количество независимых одинаково распределенных случайных величин. Однако оно возникает и в других ситуациях! Дельта-метод основан на том факте, что даже нелинейная функция от нормально распределенной случайной величины  иногда имеет распределение близкое к нормальному.

% \begin{theorem}{\textbf{Центральная Предельная Теорема (Прокопий Петрович Ляпунов)}}

% Пусть $X_1, \ldots, X_n$ попарно независимые и одинаково распределённые случайные величины с конечным вторым моментом, $0 < E(X_i^2) < \infty$, а $g(t)$ дифференцируемая функция. Пусть $\E(X_i) = \mu, \Var(X_i) = \sigma^2,$ тогда при $n \to \infty$ имеет место сходимость по распределению: 

% \[
% \sqrt{n} (g(\bar X_n) - g(\mu)) \stackrel{d}{\longrightarrow} \mN(0, \frac{\sigma^2}{n} (g'(\mu))^2 ).
% \]

% Иными словами говоря, 

% \[
% g(\bar{X}_n) \overset{\text{\textit{asy}}}{\sim} \mN \left(g(\mu), \frac{\sigma^2}{n} \cdot (g'(\mu))^2 \right).\] 
% \end{theorem}

% Попробуем применить его на практике. 

% \begin{problem}{(Равномерное)}
% Пусть случайные величины $X_1, \ldots, X_{100} \iid U[2;8]$. Как будут распределены $\bar{x}$ и $\frac{1}{\bar{x}}$? 
% \end{problem} 

% \begin{sol}
% С $\bar{x}$ всё будет просто. Воспользуемся ЦПТ, по ней 

% $$
% \bar{x}\sim \mN \left( \E(X_i), \frac{\Var(X_i)}{n} \right).
% $$

% Для равномерного распределения $\E(X_i) = \frac{2 + 8}{2} = 5$, $\Var(X_i) = \frac{(8-2)^2}{12} = 3$.

% Получается, что среднее посчитанное по сотне наблюдений будет иметь распределение 

% $$
% \bar{x}_{100} \sim \mN \left( 5, \frac{3}{100} \right).
% $$

% Для поиска распределения $\frac{1}{\bar{x}}$ воспользуемся дельта-методом: 

% $$
% g(t) = \frac{1}{t} \qquad g'(t) = -\frac{1}{t^2} \qquad g(\mu) = \frac{1}{5} \qquad g'(\mu) = - \frac{1}{25}.
% $$

% Остаётся только подставить  найденные значения в формулу и получить, что 

% $$
% \frac{1}{\bar{x}_{100}} \sim \mN \left( \frac{1}{5}, \frac{3}{100} \cdot \left(-\frac{1}{25} \right)^2\right).
% $$
% \end{sol}







% \section*{Откуда берётся дельта-метод}

% Если функция $g(t)$ дифференцируема, то в окрестности точки $\mu$ функция $g(t)$ похожа на прямую, то есть 

% $$
% g(t) \approx g(\mu) + g'(\mu) \cdot (t - \mu).
% $$

% Об этом нам говорит математический анализ, в частности, разложение в ряд Тэйлора. 

% Линейное преобразование нормально распределенной случайной величины оставляет её нормально распределенной, если угловой коэффициент отличен от нуля, т.е. 

% $$
% g'(\mu) \neq 0.
% $$ 

% Если $X \sim \mN(\mu, \sigma^2)$ и  дисперсия $X$ мала, то $X$ практически всегда попадает в небольшую окрестность $\mu$, а в ней $f$ похожа на линейную функцию и 

% $$
% g(X) \approx N(\mu, \sigma^2 (g'(\mu))^2.
% $$ 

% \indef{Получаем практическую версию дельта-метода.} Если: 

% \begin{itemize}
% 	\item  $g(t)$ --- дифференциируема;
% 	\item  $g'(\mu) \neq 0$;
% 	\item $X \sim \mN(\mu,\sigma^2)$;
% 	\item дисперсия $\sigma^2$ мала;
% \end{itemize} 

% тогда 

% $$
% g(X) \sim \mN(g(\mu),\sigma^2 (g'(\mu))^2).
% $$


% ЗБЧ позволяет использовать средние в качестве оценок для различных параметров. ЦПТ подсказывает как среднее будет распределено. Однако на практике часто встречаются ситуации, когда оценка параметра --- это функция от среднего.  \indef{Дельта-метод ---} позволяет в такой ситуации понять как будет распределена оценка. Полученное распределение можно использовать для строительства доверительного интервала. 


% \section*{Дельта-метод на практике}



% \begin{problem}{(Пуассона)}
% Пусть $X_1, \ldots, X_n \iid \Pois(\lambda)$.   С помощью дельта-метода найдите как распределена оценка вероятности $\PP(X_i = 0)$.
% \end{problem} 

% \begin{sol}
% В качестве оценки для $\lambda$ будем использовать оценку метода моментов, $\bar{x}$.  Среднее по ЦПТ имеет асимптотически нормальное распределение

% $$
% \bar{x}\sim \mN \left(\lambda, \frac{\lambda}{n} \right).
% $$

% Вероятность того, что $X_i = k$ считается по формуле 

% $$
% \PP(X_i = k) = \frac{\lambda^k}{k!} \cdot e^{-\lambda},
% $$ 

% в частности 

% $$
% \PP(X_i = 0) = e^{-\lambda}.
% $$

% Для оценки последней, $e^{-\bar{x}}$ нам нужно найти распределение. Воспользуемся  дельта-методом:

% $$
% g(t) = e^{-t} \qquad g'(t) = -e^{-t}
% $$

% Подставим значения в формулу и получим, что 

% $$
% e^{-\bar{x}} \sim \mN \left( e^{-\lambda},  \frac{\lambda}{n} \cdot e^{-2 \cdot \lambda}  \right).
% $$

% В дисперсию можем подставить вместо $\lambda$ её оценку

% $$
% e^{-\bar{x}} \sim \mN \left( e^{-\lambda},  \frac{\bar{x}}{n} \cdot e^{-2 \cdot \bar{x}}  \right).
% $$

% Такое распределение мы сможем использовать для строительства доверительных интервалов и дальнейшего анализа.
% \end{sol}


% \section*{Дельта-метод в теории}

% Естественно, строгая формулировка идеи <<дисперсия $\sigma^2$ мала>> использует понятие предела и последовательностей случайных величин.

% Если:  $g(t)$ --- дифференцируема, $g'(\mu)\neq 0$, и последовательность случайных величин $X_1, X_2, \ldots, X_n, \ldots $ удовлетворяет условию:

% \[
% \sqrt{n} (X_n - \mu) \overset{d}{\to}  \mN(0,\sigma^2),
% \]

% тогда последовательность $g(X_n)$ удовлетворяет условию:

% \[
% \sqrt{n} (g(X_n) - g(\mu)) \overset{d}{\to} \mN(0,\sigma^2 (g'(\mu))^2 )
% \]




% \subsection{Одномерный} 

% \subsection{Многомерный} 

% \section{Обобщённый метод моментов} 

% \section{Средние не панацея}

% Тут о важности предпосылок и тп на примере задачи про Киллера


% \section*{Почиташки} 

% \todo[inline]{Сюда список литературы к лекции}


\end{document}