%!TEX TS-program = xelatex
\documentclass[12pt, a4paper, oneside]{article}

% пакеты для математики
\usepackage{amsmath,amsfonts,amssymb,amsthm,mathtools}  
\mathtoolsset{showonlyrefs=true}  % Показывать номера только у тех формул, на которые есть \eqref{} в тексте.

\usepackage[british,russian]{babel} % выбор языка для документа
\usepackage[utf8]{inputenc}          % utf8 кодировка

% Основные шрифты 
\usepackage{fontspec}         
\setmainfont{Linux Libertine O}  % задаёт основной шрифт документа

% Математические шрифты 
\usepackage{unicode-math}     
\setmathfont[math-style=upright]{euler.otf} 

\setmathfont[range={\mathbb, \mathop, \heartsuit, \angle, \smile, \varheartsuit}]{Asana-Math.otf}

%%%%%%%%%% Работа с картинками и таблицами %%%%%%%%%%
\usepackage{graphicx} % Для вставки рисунков                
\usepackage{graphics}
\graphicspath{{images/}{pictures/}}   % папки с картинками

\usepackage[figurename=Картинка]{caption}

\usepackage{wrapfig}    % обтекание рисунков и таблиц текстом

\usepackage{booktabs}   % таблицы как в годных книгах
\usepackage{tabularx}   % новые типы колонок
\usepackage{tabulary}   % и ещё новые типы колонок
\usepackage{float}      % возможность позиционировать объекты в нужном месте
\renewcommand{\arraystretch}{1.2}  % больше расстояние между строками


%%%%%%%%%% Графики и рисование %%%%%%%%%%
\usepackage{tikz, pgfplots}  % языки для графики
%\pgfplotsset{compat=1.16}

\usepackage{todonotes} % для вставки в документ заметок о том, что осталось сделать
% \todo{Здесь надо коэффициенты исправить}
% \missingfigure{Здесь будет Последний день Помпеи}
% \listoftodos --- печатает все поставленные \todo'шки

\usepackage{multicol}

%%%%%%%%%% Внешний вид страницы %%%%%%%%%%

\usepackage[paper=a4paper, top=20mm, bottom=15mm,left=20mm,right=15mm]{geometry}
\usepackage{indentfirst}    % установка отступа в первом абзаце главы

\usepackage{setspace}
\setstretch{1.15}  % межстрочный интервал
\setlength{\parskip}{4mm}   % Расстояние между абзацами
% Разные длины в LaTeX: https://en.wikibooks.org/wiki/LaTeX/Lengths

% свешиваем пунктуацию
% теперь знаки пунктуации могут вылезать за правую границу текста, при этом текст выглядит ровнее
\usepackage{microtype}

% \flushbottom                            % Эта команда заставляет LaTeX чуть растягивать строки, чтобы получить идеально прямоугольную страницу
\righthyphenmin=2                       % Разрешение переноса двух и более символов
\widowpenalty=300                     % Небольшое наказание за вдовствующую строку (одна строка абзаца на этой странице, остальное --- на следующей)
\clubpenalty=3000                     % Приличное наказание за сиротствующую строку (омерзительно висящая одинокая строка в начале страницы)
\tolerance=10000     % Ещё какое-то наказание.

% мои цвета https://www.artlebedev.ru/colors/
\definecolor{titleblue}{rgb}{0.2,0.4,0.6} 
\definecolor{blue}{rgb}{0.2,0.4,0.6} 
%\definecolor{red}{rgb}{1,0,0.2} 
\definecolor{green}{rgb}{0, 0.6, 0}
\definecolor{purp}{rgb}{0.4,0,0.8} 

\definecolor{red}{RGB}{213,94,0}
\definecolor{yellow}{RGB}{240,228,66}


% цвета из geogebra 
\definecolor{litebrown}{rgb}{0.6,0.2,0}
\definecolor{darkbrown}{rgb}{0.75,0.75,0.75}

% Гиперссылки
\usepackage{xcolor}   % разные цвета

\usepackage{hyperref}
\hypersetup{
	unicode=true,           % позволяет использовать юникодные символы
	colorlinks=true,       	% true - цветные ссылки
	urlcolor=blue,          % цвет ссылки на url
	linkcolor=black,          % внутренние ссылки
	citecolor=green,        % на библиографию
	breaklinks              % если ссылка не умещается в одну строку, разбивать её на две части?
}

% меняю оформление секций 
\usepackage{titlesec}
\usepackage{sectsty}

% меняю цвет на синий
\sectionfont{\color{titleblue}}
\subsectionfont{\color{titleblue}}

% кружочки у цифр в секциях
\renewcommand{\thesection}{\arabic{section}}

% https://ru.overleaf.com/learn/latex/Sections_and_chapters

% выбрасываю нумерацию страниц и колонтитулы 
%\pagestyle{empty}

% синие круглые бульпоинты в списках itemize 
\usepackage{enumitem}

\definecolor{itemizeblue}{rgb}{0, 0.45, 0.70}

\newcommand*{\MyPoint}{\tikz \draw [baseline, fill=itemizeblue, draw=blue] circle (2.5pt);}
\renewcommand{\labelitemi}{\MyPoint}

\AddEnumerateCounter{\asbuk}{\@asbuk}{\cyrm}
\renewcommand{\theenumi}{\asbuk{enumi}}

% расстояние в списках
\setlist[itemize]{parsep=0.4em,itemsep=0em,topsep=0ex}
\setlist[enumerate]{parsep=0.4em,itemsep=0em,topsep=0ex}

% эпиграфы
\usepackage{epigraph}
\setlength\epigraphwidth{.6\textwidth}
\setlength\epigraphrule{0pt}

%%%%%%%%%% Свои команды %%%%%%%%%%

% Математические операторы первой необходимости:
\DeclareMathOperator{\sgn}{sign}
\DeclareMathOperator*{\argmin}{arg\,min}
\DeclareMathOperator*{\argmax}{arg\,max}
\DeclareMathOperator{\Cov}{Cov}
\DeclareMathOperator{\Var}{Var}
\DeclareMathOperator{\Corr}{Corr}

\DeclareMathOperator{\Pois}{Pois}
\DeclareMathOperator{\Geom}{Geom}
\DeclareMathOperator{\Exp}{Exp}

%\DeclareMathOperator{\E}{\mathbb{E}}
\DeclareMathOperator{\Med}{Med}
\DeclareMathOperator{\Mod}{Mod}
\DeclareMathOperator*{\plim}{plim}

% команды пореже
\newcommand{\const}{\mathrm{const}}  % const прямым начертанием
\newcommand{\iid}{\sim i\,i\,d\,\,}  % ну вы поняли...
\newcommand{\fr}[2]{\ensuremath{^{#1}/_{#2}}}   % особая дробь
\newcommand{\ind}[1]{\mathbbm{1}_{\{#1\}}} % Индикатор события
\newcommand{\dx}[1]{\,\mathrm{d}#1} % для интеграла: маленький отступ и прямая d

% одеваем шапки на частые штуки
\def \hb{\hat{\beta}}
\def \hs{\hat{s}}
\def \hy{\hat{y}}
\def \hY{\hat{Y}}
\def \he{\hat{\varepsilon}}
\def \hVar{\widehat{\Var}}
\def \hCorr{\widehat{\Corr}}
\def \hCov{\widehat{\Cov}}

% Греческие буквы
\def \a{\alpha}
\def \b{\beta}
\def \t{\tau}
\def \dt{\delta}
\def \e{\varepsilon}
\def \ga{\gamma}
\def \kp{\varkappa}
\def \la{\lambda}
\def \sg{\sigma}
\def \tt{\theta}
\def \Dt{\Delta}
\def \La{\Lambda}
\def \Sg{\Sigma}
\def \Tt{\Theta}
\def \Om{\Omega}
\def \om{\omega}

% Готика
\def \mA{\mathcal{A}}
\def \mB{\mathcal{B}}
\def \mC{\mathcal{C}}
\def \mE{\mathcal{E}}
\def \mF{\mathcal{F}}
\def \mH{\mathcal{H}}
\def \mL{\mathcal{L}}
\def \mN{\mathcal{N}}
\def \mU{\mathcal{U}}
\def \mV{\mathcal{V}}
\def \mW{\mathcal{W}}

% Жирные буквы
\def \mbb{\mathbb}
\def \RR{\mbb R}
\def \NN{\mbb N}
\def \ZZ{\mbb Z}
\def \PP{\mbb{P}}
\def \E{\mbb{E}}
\def \QQ{\mbb Q}

\def\F{\ensuremath{\mathcal{F}}} % аналогично!

%%%%%%%%%% Теоремы %%%%%%%%%%
\theoremstyle{plain} % Это стиль по умолчанию.  Есть другие стили.
\newtheorem{theorem}{Теорема}[section]
\newtheorem{proposition}{Утверждение}[section]
\newtheorem{result}{Следствие}[section]

% убирает курсив и что-то еще наверное делает ;)
\theoremstyle{definition}         
\newtheorem*{definition}{Определение}  % нумерация не идёт вообще


%%%%%%%%%% Задачки и решения %%%%%%%%%%
\usepackage{etoolbox}    % логические операторы для своих макросов
\usepackage{environ}
\newtoggle{lecture}

\newcounter{probNum}[section]  % счётчик для упражнений 
\NewEnviron{problem}[1]{%
    \refstepcounter{probNum}% увеличели номер на 1 
    {\noindent \textbf{\large \color{titleblue} Упражнение~\theprobNum~#1}  \\ \\ \BODY}
    {}%
  }

% Окружение, чтобы можно было убирать решения из pdf
\NewEnviron{sol}{%
  \iftoggle{lecture}
    {\noindent \textbf{\large Решение:} \\ \\ \BODY}
    {}%
  }
 
% выделение по тексту важных вещей
\newcommand{\indef}[1]{\textbf{ \color{green} #1}} 

% разные дополнения для картинок
\usetikzlibrary{arrows.meta}
\usepackage{varwidth}

\usepackage[normalem]{ulem}  % для зачекивания текста

% Если переключить в false, все solution исчезнут из pdf
\toggletrue{lecture}
%\togglefalse{lecture}




% \title{Тятя! Тятя! Наши сети притащили мертвеца!}
\title{Тятя! Тятя! Нейросети заменили продавца!}
\date{ }
\author{Ульянкин Ппилиф}

\begin{document}

% Если переключить в false, все sol исчезнут из pdf
\toggletrue{lecture}
%\togglefalse{lecture}

\maketitle

% Beware of bugs in the above code; I have only proved it correct, not tried it (Donald Knuth)
	
\begin{abstract}
    В этой виньетке собрана коллекция ручных задачек про нейросетки. Вместе с Машей можно попробовать по маленьким шажкам с ручкой и бумажкой раскрыть у себя в теле несколько чакр и немного глубже понять модели глубокого обучения\footnote{Ахахах глубже глубокого, ахахах}.
\end{abstract}

\section*{Вместо введения}

\epigraph{Я попала в сети, которые ты метил, я самая счастливая на всей планете.}{\textit{Юлианна Караулова}}

    Однажды Маша услышала про какой-то Машин лёрнинг. Она сразу же смекнула, что именно она --- та самая Маша, кому этот лёрнинг должен принадлежать. Ещё она смекнула, что если хочет владеть лёрнингом по праву, ни одна живая душа не должна сомневаться в том, что она шарит. Поэтому она постоянно изучает что-то новое. 
    
    Её друг Миша захотел стать адептом Машиного лёрнинга, и спросил её о том, как можно за вечер зашарить алгоритм обратного распространения ошибки. Тогда Маша открыла свою коллекцию учебников по глубокому обучению. В каких-то из них было написано, что ей никогда не придётся реализовывать алгоритм обратного распространения ошибки, а значит и смысла тратить время на его формулировку нет\footnote{Франсуа Шолле, Глубокое обучение на Python}. В каких-то она находила слишком сложную математику, с которой за один вечер точно не разберёшься.\footnote{Goodfellow I., Bengio Y., Courville A. Deep learning. – MIT press, 2016.} В каких-то алгоритм был описан понятно, но оставалось много недосказанностей\footnote{Николенко С., Кадурин А., Архангельская Е. Глубокое обучение. Погружение в мир нейронных сетей - Санкт-Петербург, 2018.}. 
    
    Маша решила, что для вечерних разборок нужно что-то более инфантильное. Тогда она решила поскрести по лёрнингу и собрать коллекцию ручных задачек, прорешивая которую, новые адепты Машиного лёрнинга могли бы открывать у себя диплернинговые чакры. Так и появилась эта виньетка.  
	
% \tableofcontents

% \newpage 

% \input{part_01_just_function.tex}

% \newpage 

% \input{part_02_logloss_nn_out.tex}

% \newpage 

% \input{part_03_gradient.tex}

% \newpage 

% \input{task_list_matrix_diff.tex}

\newpage 

\input{part_04_backprop.tex}

% \newpage 

% \input{part_05_lego.tex}

% \newpage 

% \input{part_06_cnn.tex}

% \newpage 

% \section{Рекурентные сетки} 

% \todo[inline]{простые задачи про RNN и LSTM}

% \todo[inline]{какие-нибудь упражнения про w2v}

% \todo[inline]{упражнение про разные модные виды ячеек типа резнетов и тп}


% \section{Итоговый тест в стиле Носко}

\end{document}









ХИ- квадрат 


\pgfplotsset{
  myplot/.style={
    width = 12cm, height = 6cm,
%     xlabel = $t$, ylabel = $f(t)$,
    samples = 75,
    domain = -5:5,
    xlabel style = {at = {(1,0)}, anchor = west},
    ylabel style = {rotate = -90, at = {(0, 1)}, anchor = south east},
  }
}

\begin{tikzpicture}[>=stealth,
  every node/.style={rounded corners},
  declare function = {
    gamma(\z) =
    (2.506628274631*sqrt(1/\z)+0.20888568*(1/\z)^(1.5)+
    0.00870357*(1/\z)^(2.5)-(174.2106599*(1/\z)^(3.5))/25920-
    (715.6423511*(1/\z)^(4.5))/1244160)*exp((-ln(1/\z)-1)*\z);
  },
  declare function = {
    gammapdf(\x,\a,\b) = (\b^\a)*\x^(\a-1)*exp(-\b*\x)/gamma(\a);
  }]

  \begin{axis}[myplot, smooth]

  \def\chisqLeft{10.117}
  \addplot[smooth, draw = none, domain = 0:\chisqLeft, fill = cyan!40] {gammapdf(x, 19/2, 0.5)} \closedcycle;
  \path[<->, draw] (axis cs: \chisqLeft, 0) to[out = 90, in = 0]
    (axis cs: \chisqLeft, 0.05) node[left] {$\chi^2_{0.1/2} = \chisqLeft$};

  \def\chisqRight{30.143}
  \addplot[smooth, draw = none, domain = \chisqRight:50, fill = orange!40] {gammapdf(x, 19/2, 0.5)} \closedcycle;
  \path[<->, draw] (axis cs: \chisqRight, 0) to[out = 90, in = 180]
    (axis cs: \chisqRight, 0.05) node[right] {$\chi^2_{1 - 0.1/2} = \chisqRight$};

  \addplot[smooth, thick, domain = 0:50, color = gray] {gammapdf(x, 19/2, 0.5)}
    node[pos = 0.38, pin = {right:$\nu = 20 - 1$}] {};

  \end{axis}
\end{tikzpicture}%







Три сигмы 

\pgfplotsset{
  myplot/.style={
    width=15cm, height=6cm,
    xlabel=$z$, ylabel=$f(z)$,
    samples=50,
    xlabel style={at={(1,0)}, anchor=west},
    ylabel style={rotate=-90, at={(0,1)}, anchor=south west},
    legend style={draw=none, fill=none},
    xmin=-4.5, xmax=4.5
  }
}%

\begin{tikzpicture}[>=stealth,
  every node/.style={rounded corners},
  declare function={
    normalpdf(\x,\mu,\sigma)=
    (2*3.1415*\sigma^2)^(-0.5)*exp(-(\x-\mu)^2/(2*\sigma^2));
  }]

  \begin{axis}[myplot, height=5cm]

    \addplot[smooth, domain=-3:3, draw=none, fill=cyan!40] {normalpdf(x, 0, 1)} \closedcycle;
    \addplot[smooth, domain=-2:2, draw=none, fill=cyan!25] {normalpdf(x, 0, 1)} \closedcycle;
    \addplot[smooth, domain=-1:1, draw=none, fill=cyan!10] {normalpdf(x, 0, 1)} \closedcycle;
    \addplot[smooth, thick, domain=-4:4] {normalpdf(x, 0, 1)};
    \addplot [ycomb, samples at={0}, color = cyan, thick] {normalpdf(x,0,1)};

  \end{axis}

  \begin{scope}[yshift=-3.5cm]
    \begin{axis}[
      myplot,
      hide y axis,
      height=4.5cm,
      axis x line*=bottom,
      xlabel = ,
      xtick = {-3, -2, -1, 0, 1, 2, 3},
      xticklabels = {$\mu -3\sigma$, $\mu - 2\sigma$, $\mu -1\sigma$, $\mu$, $\mu + 1\sigma$, $\mu + 2\sigma$, $\mu + 3\sigma$},
      domain=-4:4, ymin = -0.05, ymax = 0.3]

      \addplot[draw=none] {x};
    \draw[|<->|] (axis cs: -1, 0.20) -- (axis cs: 1, 0.20) node [above, midway] {$\text{P}(-1 < Z < 1) \approx 0.6826$};
    \draw[|<->|] (axis cs: -2, 0.10) -- (axis cs: 2, 0.10) node [above, midway] {$\text{P}(-2 < Z < 2) \approx 0.9546$};
    \draw[|<->|] (axis cs: -3, 0.00) -- (axis cs: 3, 0.00) node [above, midway] {$\text{P}(-3 < Z < 3) \approx 0.9973$};

    \end{axis}
  \end{scope}

\end{tikzpicture}%


Для степеней свободы

"Они могут отнять у нас наши жизни, но они никогда не отнимут нашу свободу!"
"Храброе сердце", 1995


Открою тебе маленький секрет. Каждый день, раз в день, делай себе маленький подарок. Не планируй заранее, не жди его, просто пусть он случается. Это может быть новая рубашка или послеобеденный сон в кабинете, или две чашки хорошего, горячего, черного кофе.

Дейл Купер