%!TEX TS-program = xelatex
\documentclass[12pt, a4paper, oneside]{article}

% Можно вставить разную преамбулу
% пакеты для математики
\usepackage{amsmath,amsfonts,amssymb,amsthm,mathtools}  
\mathtoolsset{showonlyrefs=true}  % Показывать номера только у тех формул, на которые есть \eqref{} в тексте.

\usepackage[british,russian]{babel} % выбор языка для документа
\usepackage[utf8]{inputenc}          % utf8 кодировка

% Основные шрифты 
\usepackage{fontspec}         
\setmainfont{Linux Libertine O}  % задаёт основной шрифт документа

% Математические шрифты 
\usepackage{unicode-math}     
\setmathfont[math-style=upright]{euler.otf} 

\setmathfont[range={\mathbb, \mathop, \heartsuit, \angle, \smile, \varheartsuit}]{Asana-Math.otf}

%%%%%%%%%% Работа с картинками и таблицами %%%%%%%%%%
\usepackage{graphicx} % Для вставки рисунков                
\usepackage{graphics}
\graphicspath{{images/}{pictures/}}   % папки с картинками

\usepackage[figurename=Картинка]{caption}

\usepackage{wrapfig}    % обтекание рисунков и таблиц текстом

\usepackage{booktabs}   % таблицы как в годных книгах
\usepackage{tabularx}   % новые типы колонок
\usepackage{tabulary}   % и ещё новые типы колонок
\usepackage{float}      % возможность позиционировать объекты в нужном месте
\renewcommand{\arraystretch}{1.2}  % больше расстояние между строками


%%%%%%%%%% Графики и рисование %%%%%%%%%%
\usepackage{tikz, pgfplots}  % языки для графики
%\pgfplotsset{compat=1.16}

\usepackage{todonotes} % для вставки в документ заметок о том, что осталось сделать
% \todo{Здесь надо коэффициенты исправить}
% \missingfigure{Здесь будет Последний день Помпеи}
% \listoftodos --- печатает все поставленные \todo'шки

\usepackage{multicol}

%%%%%%%%%% Внешний вид страницы %%%%%%%%%%

\usepackage[paper=a4paper, top=20mm, bottom=15mm,left=20mm,right=15mm]{geometry}
\usepackage{indentfirst}    % установка отступа в первом абзаце главы

\usepackage{setspace}
\setstretch{1.15}  % межстрочный интервал
\setlength{\parskip}{4mm}   % Расстояние между абзацами
% Разные длины в LaTeX: https://en.wikibooks.org/wiki/LaTeX/Lengths

% свешиваем пунктуацию
% теперь знаки пунктуации могут вылезать за правую границу текста, при этом текст выглядит ровнее
\usepackage{microtype}

% \flushbottom                            % Эта команда заставляет LaTeX чуть растягивать строки, чтобы получить идеально прямоугольную страницу
\righthyphenmin=2                       % Разрешение переноса двух и более символов
\widowpenalty=300                     % Небольшое наказание за вдовствующую строку (одна строка абзаца на этой странице, остальное --- на следующей)
\clubpenalty=3000                     % Приличное наказание за сиротствующую строку (омерзительно висящая одинокая строка в начале страницы)
\tolerance=10000     % Ещё какое-то наказание.

% мои цвета https://www.artlebedev.ru/colors/
\definecolor{titleblue}{rgb}{0.2,0.4,0.6} 
\definecolor{blue}{rgb}{0.2,0.4,0.6} 
%\definecolor{red}{rgb}{1,0,0.2} 
\definecolor{green}{rgb}{0, 0.6, 0}
\definecolor{purp}{rgb}{0.4,0,0.8} 

\definecolor{red}{RGB}{213,94,0}
\definecolor{yellow}{RGB}{240,228,66}


% цвета из geogebra 
\definecolor{litebrown}{rgb}{0.6,0.2,0}
\definecolor{darkbrown}{rgb}{0.75,0.75,0.75}

% Гиперссылки
\usepackage{xcolor}   % разные цвета

\usepackage{hyperref}
\hypersetup{
	unicode=true,           % позволяет использовать юникодные символы
	colorlinks=true,       	% true - цветные ссылки
	urlcolor=blue,          % цвет ссылки на url
	linkcolor=black,          % внутренние ссылки
	citecolor=green,        % на библиографию
	breaklinks              % если ссылка не умещается в одну строку, разбивать её на две части?
}

% меняю оформление секций 
\usepackage{titlesec}
\usepackage{sectsty}

% меняю цвет на синий
\sectionfont{\color{titleblue}}
\subsectionfont{\color{titleblue}}

% кружочки у цифр в секциях
\renewcommand{\thesection}{\arabic{section}}

% https://ru.overleaf.com/learn/latex/Sections_and_chapters

% выбрасываю нумерацию страниц и колонтитулы 
%\pagestyle{empty}

% синие круглые бульпоинты в списках itemize 
\usepackage{enumitem}

\definecolor{itemizeblue}{rgb}{0, 0.45, 0.70}

\newcommand*{\MyPoint}{\tikz \draw [baseline, fill=itemizeblue, draw=blue] circle (2.5pt);}
\renewcommand{\labelitemi}{\MyPoint}

\AddEnumerateCounter{\asbuk}{\@asbuk}{\cyrm}
\renewcommand{\theenumi}{\asbuk{enumi}}

% расстояние в списках
\setlist[itemize]{parsep=0.4em,itemsep=0em,topsep=0ex}
\setlist[enumerate]{parsep=0.4em,itemsep=0em,topsep=0ex}

% эпиграфы
\usepackage{epigraph}
\setlength\epigraphwidth{.6\textwidth}
\setlength\epigraphrule{0pt}

%%%%%%%%%% Свои команды %%%%%%%%%%

% Математические операторы первой необходимости:
\DeclareMathOperator{\sgn}{sign}
\DeclareMathOperator*{\argmin}{arg\,min}
\DeclareMathOperator*{\argmax}{arg\,max}
\DeclareMathOperator{\Cov}{Cov}
\DeclareMathOperator{\Var}{Var}
\DeclareMathOperator{\Corr}{Corr}

\DeclareMathOperator{\Pois}{Pois}
\DeclareMathOperator{\Geom}{Geom}
\DeclareMathOperator{\Exp}{Exp}

%\DeclareMathOperator{\E}{\mathbb{E}}
\DeclareMathOperator{\Med}{Med}
\DeclareMathOperator{\Mod}{Mod}
\DeclareMathOperator*{\plim}{plim}

% команды пореже
\newcommand{\const}{\mathrm{const}}  % const прямым начертанием
\newcommand{\iid}{\sim i\,i\,d\,\,}  % ну вы поняли...
\newcommand{\fr}[2]{\ensuremath{^{#1}/_{#2}}}   % особая дробь
\newcommand{\ind}[1]{\mathbbm{1}_{\{#1\}}} % Индикатор события
\newcommand{\dx}[1]{\,\mathrm{d}#1} % для интеграла: маленький отступ и прямая d

% одеваем шапки на частые штуки
\def \hb{\hat{\beta}}
\def \hs{\hat{s}}
\def \hy{\hat{y}}
\def \hY{\hat{Y}}
\def \he{\hat{\varepsilon}}
\def \hVar{\widehat{\Var}}
\def \hCorr{\widehat{\Corr}}
\def \hCov{\widehat{\Cov}}

% Греческие буквы
\def \a{\alpha}
\def \b{\beta}
\def \t{\tau}
\def \dt{\delta}
\def \e{\varepsilon}
\def \ga{\gamma}
\def \kp{\varkappa}
\def \la{\lambda}
\def \sg{\sigma}
\def \tt{\theta}
\def \Dt{\Delta}
\def \La{\Lambda}
\def \Sg{\Sigma}
\def \Tt{\Theta}
\def \Om{\Omega}
\def \om{\omega}

% Готика
\def \mA{\mathcal{A}}
\def \mB{\mathcal{B}}
\def \mC{\mathcal{C}}
\def \mE{\mathcal{E}}
\def \mF{\mathcal{F}}
\def \mH{\mathcal{H}}
\def \mL{\mathcal{L}}
\def \mN{\mathcal{N}}
\def \mU{\mathcal{U}}
\def \mV{\mathcal{V}}
\def \mW{\mathcal{W}}

% Жирные буквы
\def \mbb{\mathbb}
\def \RR{\mbb R}
\def \NN{\mbb N}
\def \ZZ{\mbb Z}
\def \PP{\mbb{P}}
\def \E{\mbb{E}}
\def \QQ{\mbb Q}

\def\F{\ensuremath{\mathcal{F}}} % аналогично!

%%%%%%%%%% Теоремы %%%%%%%%%%
\theoremstyle{plain} % Это стиль по умолчанию.  Есть другие стили.
\newtheorem{theorem}{Теорема}[section]
\newtheorem{proposition}{Утверждение}[section]
\newtheorem{result}{Следствие}[section]

% убирает курсив и что-то еще наверное делает ;)
\theoremstyle{definition}         
\newtheorem*{definition}{Определение}  % нумерация не идёт вообще


%%%%%%%%%% Задачки и решения %%%%%%%%%%
\usepackage{etoolbox}    % логические операторы для своих макросов
\usepackage{environ}
\newtoggle{lecture}

\newcounter{probNum}[section]  % счётчик для упражнений 
\NewEnviron{problem}[1]{%
    \refstepcounter{probNum}% увеличели номер на 1 
    {\noindent \textbf{\large \color{titleblue} Упражнение~\theprobNum~#1}  \\ \\ \BODY}
    {}%
  }

% Окружение, чтобы можно было убирать решения из pdf
\NewEnviron{sol}{%
  \iftoggle{lecture}
    {\noindent \textbf{\large Решение:} \\ \\ \BODY}
    {}%
  }
 
% выделение по тексту важных вещей
\newcommand{\indef}[1]{\textbf{ \color{green} #1}} 

% разные дополнения для картинок
\usetikzlibrary{arrows.meta}
\usepackage{varwidth}

\usepackage[normalem]{ulem}  % для зачекивания текста

% Если переключить в false, все solution исчезнут из pdf
\toggletrue{lecture}
%\togglefalse{lecture}



\title{
\begin{center} 
\includegraphics[width=0.99\textwidth]{logo.png}
\end{center}

Посиделка 8: мощь средних}
\date{ } %\today}

% Если делаешь конспект, вписывай своё имя прямо сюда!
\author{Ульянкин Ппилиф \thanks{\url{https://github.com/FUlyankin/matstat_lec}}}

\begin{document} % Конец преамбулы, начало файла

\maketitle

\epigraph{\hfill Бесконечность --- не предел!}{\textit{Баз Лайтер (История игрушек, 1995)}}

В прошлой посиделке мы посмотрели на то, как ЦПТ и ЗБЧ позволяют нам оценивать неизвестные параметры, строить для них доверительные интервалы и проверять гипотезы. В этой посиделки мы продолжим эту линию и поговорим про то, как можно сконструировать АБ-тест для долей и средних с помощью асимптотического подхода.

\section{Что такое АБ-тестирование}

\todo[inline]{Красивое описание проблемы. О бизнесе, о том что каждый день тестируют убере и тп. }

% Есть Винни-Пух и он торгует мёдом. Винни-пух решил проверить что произойдет в его онлайн-магазине с покупками, если он сделает редизайн сайта. Редизайн ему подсказали UX-исследователи. Теперь ВП хочет проверить что будет с конверсиями пользователей в покупки. 

% - Treatment: $p_T$ - ей он показывает новый сайт $(5\% от всех пользователей)$
% - Conrol: $p_C$ - ей он показывает старый сайт 

% $H_0: p_T - p_C = 0$

% $H_A: p_T - p_C > 0$


\section{ }


% __1. Эксперимент__ 

% Предпосылки теста:

% - Мы должны отдавать себе отчёт, что все наблюдения должны быть независимы и одинаково распределены, чтобы тест на основе ЦПТ работал.
% - Много наблюдений, мы работаем с биг-датой.
% - У наших наблюдений конечная дисперсия (ни одно из наблюдений особо сильно не выделяется на фоне всех остальных). Иначе говоря у нас нес выбросов в данных. 



% __2. Союзник:__



% $$
% \bar X_n \overset{asy}{\sim} N \left( \mathbb{E}(X_i), \frac{Var(X_i)}{n} \right)
% $$

% __3. Данные и модель:__

% Две выборки независят друг от друга, так как пользователь случайно относится к одной из двух групп. 

% \begin{equation*}
%     \begin{aligned}
%          & X^c_1, \ldots, X^c_{n_c} \sim idd \quad Bern(p_c) \\
%          & X^T_1, \ldots, X^T_{n_T} \sim idd \quad Bern(p_T) 
%     \end{aligned}
% \end{equation*}

% __4. Критерий для проверки__

% С помощью ЦПТ я могу выписать следующую логику: 

% \begin{equation*}
%     \begin{aligned}
%         & \bar X_{n_c} = \hat p_c \overset{asy}{\sim} N \left( p_c, \frac{ p_c\cdot (1 - p_c)}{n_c} \right) \approx  N \left( p_c, \frac{ \hat p_c\cdot (1 - \hat p_c)}{n_c} \right)  \\
%         & \bar X_{n_T} = \hat p_T \overset{asy}{\sim} N \left( p_T, \frac{ p_T\cdot (1 - p_T)}{n_T} \right) \approx N \left( p_T, \frac{ \hat p_T\cdot (1 - \hat p_T)}{n_T} \right)
%     \end{aligned}
% \end{equation*}


% $$
% \hat p_c - \hat p_T \overset{asy}{\sim} N \left( p_c - p_T, \frac{ \hat p_c\cdot (1 - \hat p_c)}{n_c} + \frac{ \hat p_T \cdot (1 - \hat p_T)}{n_T}  \right)
% $$


% А дальше мы говорим: "А пусть у нас верна нулевая гипотеза, мы верим в статус-кво и нам нужны доказательства что он нарушен".

% $$
% \hat p_c - \hat p_T \underset{H_0}{\overset{asy}{\sim}} N \left(0, \frac{ \hat p_c\cdot (1 - \hat p_c)}{n_c} + \frac{ \hat p_T \cdot (1 - \hat p_T)}{n_T}  \right)
% $$

% $$
% Z = \frac{\hat p_c - \hat p_T}{ \sqrt{ \frac{\hat p_c\cdot (1 - \hat p_c)}{n_c} + \frac{ \hat p_T \cdot (1 - \hat p_T)}{n_T}}} \underset{H_0}{\overset{asy}{\sim}} N (0, 1)
% $$

% Ечли у нас наблюдаемое значение $z_{obs}$ оказывается в хвосте распределения, это означает что расстоение между долями очень большое. Видимо, разница между ними действительно есть. 

% Понятное дело, что по анлогии это делается для любых средних! То есть вы можете попробовать предположить, что выборка пришла из какого-то другого распределения, понять как выглядит дисперсия и подготовить асимптотический критерий. 


% > Остаётся много вопросиков про то, а как правильно спланировать эксперимент. Например, сколько надо наблюдений? 

% # Планирование эксперимента



% Нам надо как-то увязать между собой следующие показатели: 

% - MDE (Minimal detectable effect)
% - ошибка 1 рода
% - ошибка 2 рода 
% - количество наблюдений $n_c,n_T$
% - как должны между собой соотноситься $n_c$ и $n_T$



% __Способы выбрать $q$__

% Наш бюджет на эксперимент всегда ограничен. Мы всего можем собрать $n$ наблюдений, из которых в группу воздействия попадёт $q \cdot n$, то есть $n_T = q \cdot n$ и $n_c = (1 - q) \cdot n$. Как выбрать $q$? 




% __1)__  От балды: $q = 0.5$.


% __2)__  Хочется, чтобы дисперсия была поменьше, то есть: 


% $$
% Var(\hat p_T - \hat p_c) = \frac{p_T \cdot (1 - p_T)}{q \cdot n} + \frac{p_c \cdot (1 - p_c)}{(1 - q) \cdot n}  \to \min_q
% $$

% Если мы решим эту задачу, тогда у нас поулчится

% $$
% q = \frac{\sqrt{p_T \cdot (1 - p_T)} }{\sqrt{p_T \cdot (1 - p_T)} + \sqrt{p_c \cdot (1 - p_c)} }
% $$


% Скорее всего, все революционные идеи уже сделаны. Скорее всего, наши группы в целом похожи. $MDE$ маленькое. То есть $q \approx 0.5.$ 

% Либо если вы верите, что ваши изменения революционные и перевернут мир, можно запустить дорогой двухшаговый эксперимент, где за первую неделю вы получите точечную оценку для $q$, а на второй неделе с выверенной пропорцией уже проверите гипотезу. 

% __3)__ Обычно есть продукт сам по себе. Обычно для АБ-теста от этого продукта отсекают маленькую долю людей, например $1\%$. Нужно просто понять хватает ли нам для детекции эффекта $1\%$ или надо увеличить долю. 

% __4)__  Выбрать $q$ так, чтобы минимизировать $\beta$. Ниже мы подумали над этой идеей и она совпала с идеей номер 2. 



% Давайте проведёмс вычисления, которые свяжут эти три показателя для ситуации, когда $q = 0.5$.

% Критерий для двусторонней альтернативы: 

% $$
% \left| \frac{\hat d - 0}{se(\hat d)} \right| > z_{1-\frac{\alpha}{2}} \quad \Rightarrow \quad \text{отвергаем } H_0
% $$



\end{document}
