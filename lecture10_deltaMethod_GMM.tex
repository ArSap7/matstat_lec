%!TEX TS-program = xelatex
\documentclass[12pt, a4paper, oneside]{article}

% Можно вставить разную преамбулу
% пакеты для математики
\usepackage{amsmath,amsfonts,amssymb,amsthm,mathtools}  
\mathtoolsset{showonlyrefs=true}  % Показывать номера только у тех формул, на которые есть \eqref{} в тексте.

\usepackage[british,russian]{babel} % выбор языка для документа
\usepackage[utf8]{inputenc}          % utf8 кодировка

% Основные шрифты 
\usepackage{fontspec}         
\setmainfont{Linux Libertine O}  % задаёт основной шрифт документа

% Математические шрифты 
\usepackage{unicode-math}     
\setmathfont[math-style=upright]{euler.otf} 

\setmathfont[range={\mathbb, \mathop, \heartsuit, \angle, \smile, \varheartsuit}]{Asana-Math.otf}

%%%%%%%%%% Работа с картинками и таблицами %%%%%%%%%%
\usepackage{graphicx} % Для вставки рисунков                
\usepackage{graphics}
\graphicspath{{images/}{pictures/}}   % папки с картинками

\usepackage[figurename=Картинка]{caption}

\usepackage{wrapfig}    % обтекание рисунков и таблиц текстом

\usepackage{booktabs}   % таблицы как в годных книгах
\usepackage{tabularx}   % новые типы колонок
\usepackage{tabulary}   % и ещё новые типы колонок
\usepackage{float}      % возможность позиционировать объекты в нужном месте
\renewcommand{\arraystretch}{1.2}  % больше расстояние между строками


%%%%%%%%%% Графики и рисование %%%%%%%%%%
\usepackage{tikz, pgfplots}  % языки для графики
%\pgfplotsset{compat=1.16}

\usepackage{todonotes} % для вставки в документ заметок о том, что осталось сделать
% \todo{Здесь надо коэффициенты исправить}
% \missingfigure{Здесь будет Последний день Помпеи}
% \listoftodos --- печатает все поставленные \todo'шки

\usepackage{multicol}

%%%%%%%%%% Внешний вид страницы %%%%%%%%%%

\usepackage[paper=a4paper, top=20mm, bottom=15mm,left=20mm,right=15mm]{geometry}
\usepackage{indentfirst}    % установка отступа в первом абзаце главы

\usepackage{setspace}
\setstretch{1.15}  % межстрочный интервал
\setlength{\parskip}{4mm}   % Расстояние между абзацами
% Разные длины в LaTeX: https://en.wikibooks.org/wiki/LaTeX/Lengths

% свешиваем пунктуацию
% теперь знаки пунктуации могут вылезать за правую границу текста, при этом текст выглядит ровнее
\usepackage{microtype}

% \flushbottom                            % Эта команда заставляет LaTeX чуть растягивать строки, чтобы получить идеально прямоугольную страницу
\righthyphenmin=2                       % Разрешение переноса двух и более символов
\widowpenalty=300                     % Небольшое наказание за вдовствующую строку (одна строка абзаца на этой странице, остальное --- на следующей)
\clubpenalty=3000                     % Приличное наказание за сиротствующую строку (омерзительно висящая одинокая строка в начале страницы)
\tolerance=10000     % Ещё какое-то наказание.

% мои цвета https://www.artlebedev.ru/colors/
\definecolor{titleblue}{rgb}{0.2,0.4,0.6} 
\definecolor{blue}{rgb}{0.2,0.4,0.6} 
%\definecolor{red}{rgb}{1,0,0.2} 
\definecolor{green}{rgb}{0, 0.6, 0}
\definecolor{purp}{rgb}{0.4,0,0.8} 

\definecolor{red}{RGB}{213,94,0}
\definecolor{yellow}{RGB}{240,228,66}


% цвета из geogebra 
\definecolor{litebrown}{rgb}{0.6,0.2,0}
\definecolor{darkbrown}{rgb}{0.75,0.75,0.75}

% Гиперссылки
\usepackage{xcolor}   % разные цвета

\usepackage{hyperref}
\hypersetup{
	unicode=true,           % позволяет использовать юникодные символы
	colorlinks=true,       	% true - цветные ссылки
	urlcolor=blue,          % цвет ссылки на url
	linkcolor=black,          % внутренние ссылки
	citecolor=green,        % на библиографию
	breaklinks              % если ссылка не умещается в одну строку, разбивать её на две части?
}

% меняю оформление секций 
\usepackage{titlesec}
\usepackage{sectsty}

% меняю цвет на синий
\sectionfont{\color{titleblue}}
\subsectionfont{\color{titleblue}}

% кружочки у цифр в секциях
\renewcommand{\thesection}{\arabic{section}}

% https://ru.overleaf.com/learn/latex/Sections_and_chapters

% выбрасываю нумерацию страниц и колонтитулы 
%\pagestyle{empty}

% синие круглые бульпоинты в списках itemize 
\usepackage{enumitem}

\definecolor{itemizeblue}{rgb}{0, 0.45, 0.70}

\newcommand*{\MyPoint}{\tikz \draw [baseline, fill=itemizeblue, draw=blue] circle (2.5pt);}
\renewcommand{\labelitemi}{\MyPoint}

\AddEnumerateCounter{\asbuk}{\@asbuk}{\cyrm}
\renewcommand{\theenumi}{\asbuk{enumi}}

% расстояние в списках
\setlist[itemize]{parsep=0.4em,itemsep=0em,topsep=0ex}
\setlist[enumerate]{parsep=0.4em,itemsep=0em,topsep=0ex}

% эпиграфы
\usepackage{epigraph}
\setlength\epigraphwidth{.6\textwidth}
\setlength\epigraphrule{0pt}

%%%%%%%%%% Свои команды %%%%%%%%%%

% Математические операторы первой необходимости:
\DeclareMathOperator{\sgn}{sign}
\DeclareMathOperator*{\argmin}{arg\,min}
\DeclareMathOperator*{\argmax}{arg\,max}
\DeclareMathOperator{\Cov}{Cov}
\DeclareMathOperator{\Var}{Var}
\DeclareMathOperator{\Corr}{Corr}

\DeclareMathOperator{\Pois}{Pois}
\DeclareMathOperator{\Geom}{Geom}
\DeclareMathOperator{\Exp}{Exp}

%\DeclareMathOperator{\E}{\mathbb{E}}
\DeclareMathOperator{\Med}{Med}
\DeclareMathOperator{\Mod}{Mod}
\DeclareMathOperator*{\plim}{plim}

% команды пореже
\newcommand{\const}{\mathrm{const}}  % const прямым начертанием
\newcommand{\iid}{\sim i\,i\,d\,\,}  % ну вы поняли...
\newcommand{\fr}[2]{\ensuremath{^{#1}/_{#2}}}   % особая дробь
\newcommand{\ind}[1]{\mathbbm{1}_{\{#1\}}} % Индикатор события
\newcommand{\dx}[1]{\,\mathrm{d}#1} % для интеграла: маленький отступ и прямая d

% одеваем шапки на частые штуки
\def \hb{\hat{\beta}}
\def \hs{\hat{s}}
\def \hy{\hat{y}}
\def \hY{\hat{Y}}
\def \he{\hat{\varepsilon}}
\def \hVar{\widehat{\Var}}
\def \hCorr{\widehat{\Corr}}
\def \hCov{\widehat{\Cov}}

% Греческие буквы
\def \a{\alpha}
\def \b{\beta}
\def \t{\tau}
\def \dt{\delta}
\def \e{\varepsilon}
\def \ga{\gamma}
\def \kp{\varkappa}
\def \la{\lambda}
\def \sg{\sigma}
\def \tt{\theta}
\def \Dt{\Delta}
\def \La{\Lambda}
\def \Sg{\Sigma}
\def \Tt{\Theta}
\def \Om{\Omega}
\def \om{\omega}

% Готика
\def \mA{\mathcal{A}}
\def \mB{\mathcal{B}}
\def \mC{\mathcal{C}}
\def \mE{\mathcal{E}}
\def \mF{\mathcal{F}}
\def \mH{\mathcal{H}}
\def \mL{\mathcal{L}}
\def \mN{\mathcal{N}}
\def \mU{\mathcal{U}}
\def \mV{\mathcal{V}}
\def \mW{\mathcal{W}}

% Жирные буквы
\def \mbb{\mathbb}
\def \RR{\mbb R}
\def \NN{\mbb N}
\def \ZZ{\mbb Z}
\def \PP{\mbb{P}}
\def \E{\mbb{E}}
\def \QQ{\mbb Q}

\def\F{\ensuremath{\mathcal{F}}} % аналогично!

%%%%%%%%%% Теоремы %%%%%%%%%%
\theoremstyle{plain} % Это стиль по умолчанию.  Есть другие стили.
\newtheorem{theorem}{Теорема}[section]
\newtheorem{proposition}{Утверждение}[section]
\newtheorem{result}{Следствие}[section]

% убирает курсив и что-то еще наверное делает ;)
\theoremstyle{definition}         
\newtheorem*{definition}{Определение}  % нумерация не идёт вообще


%%%%%%%%%% Задачки и решения %%%%%%%%%%
\usepackage{etoolbox}    % логические операторы для своих макросов
\usepackage{environ}
\newtoggle{lecture}

\newcounter{probNum}[section]  % счётчик для упражнений 
\NewEnviron{problem}[1]{%
    \refstepcounter{probNum}% увеличели номер на 1 
    {\noindent \textbf{\large \color{titleblue} Упражнение~\theprobNum~#1}  \\ \\ \BODY}
    {}%
  }

% Окружение, чтобы можно было убирать решения из pdf
\NewEnviron{sol}{%
  \iftoggle{lecture}
    {\noindent \textbf{\large Решение:} \\ \\ \BODY}
    {}%
  }
 
% выделение по тексту важных вещей
\newcommand{\indef}[1]{\textbf{ \color{green} #1}} 

% разные дополнения для картинок
\usetikzlibrary{arrows.meta}
\usepackage{varwidth}

\usepackage[normalem]{ulem}  % для зачекивания текста

% Если переключить в false, все solution исчезнут из pdf
\toggletrue{lecture}
%\togglefalse{lecture}



\title{
\begin{center} 
\includegraphics[width=0.99\textwidth]{logo.png}
\end{center}

Посиделка 10: дельта-метод и обобщенный метод моментов}
\date{ } %\today}

% Если делаешь конспект, вписывай своё имя прямо сюда!
\author{Ульянкин Ппилиф \thanks{\url{https://github.com/FUlyankin/matstat_lec}}}

\begin{document} % Конец преамбулы, начало файла

\maketitle

\epigraph{\hfill Цитата}{\textit{Автор (Откуда, гол)}}

Когда оценка моментов в явном виде выражается через среднее, работать с ним довольно легко. В посиделке про мощь средних мы решали задачу про киндеры и геометрическое распределение. Там у нас получилась оценка $\hat{p} = \frac{1}{\bar{X}_n}.$ Получить распределение оценки с помощью ЦПТ не выйдет. Придётся обратиться к её обобщению,  \indef{дельта-методу.}

\section{Дельта-метод}

Нормальное распределение возникает, если суммируется большое количество независимых одинаково распределенных случайных величин. Однако оно возникает и в других ситуациях! Дельта-метод основан на том факте, что даже нелинейная функция от нормально распределенной случайной величины  иногда имеет распределение близкое к нормальному.

\begin{theorem}{\textbf{Центральная Предельная Теорема (Прокопий Петрович Ляпунов)}}

Пусть $X_1, \ldots, X_n$ попарно независимые и одинаково распределённые случайные величины с конечным вторым моментом, $0 < E(X_i^2) < \infty$, а $g(t)$ дифференцируемая функция. Пусть $\E(X_i) = \mu, \Var(X_i) = \sigma^2,$ тогда при $n \to \infty$ имеет место сходимость по распределению: 

\[
\sqrt{n} (g(\bar X_n) - g(\mu)) \stackrel{d}{\longrightarrow} \mN(0, \frac{\sigma^2}{n} (g'(\mu))^2 ).
\]

Иными словами говоря, 

\[
g(\bar{X}_n) \overset{\text{\textit{asy}}}{\sim} \mN \left(g(\mu), \frac{\sigma^2}{n} \cdot (g'(\mu))^2 \right).\] 
\end{theorem}

Попробуем применить его на практике. 

\begin{problem}{(Равномерное)}
Пусть случайные величины $X_1, \ldots, X_{100} \iid U[2;8]$. Как будут распределены $\bar{x}$ и $\frac{1}{\bar{x}}$? 
\end{problem} 

\begin{sol}
С $\bar{x}$ всё будет просто. Воспользуемся ЦПТ, по ней 

$$
\bar{x}\sim \mN \left( \E(X_i), \frac{\Var(X_i)}{n} \right).
$$

Для равномерного распределения $\E(X_i) = \frac{2 + 8}{2} = 5$, $\Var(X_i) = \frac{(8-2)^2}{12} = 3$.

Получается, что среднее посчитанное по сотне наблюдений будет иметь распределение 

$$
\bar{x}_{100} \sim \mN \left( 5, \frac{3}{100} \right).
$$

Для поиска распределения $\frac{1}{\bar{x}}$ воспользуемся дельта-методом: 

$$
g(t) = \frac{1}{t} \qquad g'(t) = -\frac{1}{t^2} \qquad g(\mu) = \frac{1}{5} \qquad g'(\mu) = - \frac{1}{25}.
$$

Остаётся только подставить  найденные значения в формулу и получить, что 

$$
\frac{1}{\bar{x}_{100}} \sim \mN \left( \frac{1}{5}, \frac{3}{100} \cdot \left(-\frac{1}{25} \right)^2\right).
$$
\end{sol}


\section*{Откуда берётся дельта-метод}

Если функция $g(t)$ дифференцируема, то в окрестности точки $\mu$ функция $g(t)$ похожа на прямую, то есть 

$$
g(t) \approx g(\mu) + g'(\mu) \cdot (t - \mu).
$$

Об этом нам говорит математический анализ, в частности, разложение в ряд Тэйлора. 

Линейное преобразование нормально распределенной случайной величины оставляет её нормально распределенной, если угловой коэффициент отличен от нуля, т.е. 

$$
g'(\mu) \neq 0.
$$ 

Если $X \sim \mN(\mu, \sigma^2)$ и  дисперсия $X$ мала, то $X$ практически всегда попадает в небольшую окрестность $\mu$, а в ней $f$ похожа на линейную функцию и 

$$
g(X) \approx N(\mu, \sigma^2 (g'(\mu))^2.
$$ 

\indef{Получаем практическую версию дельта-метода.} Если: 

\begin{itemize}
	\item  $g(t)$ --- дифференциируема;
	\item  $g'(\mu) \neq 0$;
	\item $X \sim \mN(\mu,\sigma^2)$;
	\item дисперсия $\sigma^2$ мала;
\end{itemize} 

тогда 

$$
g(X) \sim \mN(g(\mu),\sigma^2 (g'(\mu))^2).
$$


ЗБЧ позволяет использовать средние в качестве оценок для различных параметров. ЦПТ подсказывает как среднее будет распределено. Однако на практике часто встречаются ситуации, когда оценка параметра --- это функция от среднего.  \indef{Дельта-метод ---} позволяет в такой ситуации понять как будет распределена оценка. Полученное распределение можно использовать для строительства доверительного интервала. 


\section*{Дельта-метод на практике}


\begin{problem}{(Пуассона)}
Пусть $X_1, \ldots, X_n \iid \Pois(\lambda)$.   С помощью дельта-метода найдите как распределена оценка вероятности $\PP(X_i = 0)$.
\end{problem} 

\begin{sol}
В качестве оценки для $\lambda$ будем использовать оценку метода моментов, $\bar{x}$.  Среднее по ЦПТ имеет асимптотически нормальное распределение

$$
\bar{x}\sim \mN \left(\lambda, \frac{\lambda}{n} \right).
$$

Вероятность того, что $X_i = k$ считается по формуле 

$$
\PP(X_i = k) = \frac{\lambda^k}{k!} \cdot e^{-\lambda},
$$ 

в частности 

$$
\PP(X_i = 0) = e^{-\lambda}.
$$

Для оценки последней, $e^{-\bar{x}}$ нам нужно найти распределение. Воспользуемся  дельта-методом:

$$
g(t) = e^{-t} \qquad g'(t) = -e^{-t}
$$

Подставим значения в формулу и получим, что 

$$
e^{-\bar{x}} \sim \mN \left( e^{-\lambda},  \frac{\lambda}{n} \cdot e^{-2 \cdot \lambda}  \right).
$$

В дисперсию можем подставить вместо $\lambda$ её оценку

$$
e^{-\bar{x}} \sim \mN \left( e^{-\lambda},  \frac{\bar{x}}{n} \cdot e^{-2 \cdot \bar{x}}  \right).
$$

Такое распределение мы сможем использовать для строительства доверительных интервалов и дальнейшего анализа.
\end{sol}


\section*{Дельта-метод в теории}

Естественно, строгая формулировка идеи <<дисперсия $\sigma^2$ мала>> использует понятие предела и последовательностей случайных величин.

Если:  $g(t)$ --- дифференцируема, $g'(\mu)\neq 0$, и последовательность случайных величин $X_1, X_2, \ldots, X_n, \ldots $ удовлетворяет условию:

\[
\sqrt{n} (X_n - \mu) \overset{d}{\to}  \mN(0,\sigma^2),
\]

тогда последовательность $g(X_n)$ удовлетворяет условию:

\[
\sqrt{n} (g(X_n) - g(\mu)) \overset{d}{\to} \mN(0,\sigma^2 (g'(\mu))^2 )
\]


\subsection{Одномерный} 

\subsection{Многомерный} 

\section{Обобщённый метод моментов} 

\section{Средние не панацея}

Тут о важности предпосылок и тп на примере задачи про Киллера


\section*{Почиташки} 

\todo[inline]{Сюда список литературы к лекции}


\end{document}


